\documentclass[color]{tongjithesis}
\addbibresource[location=local]{tongjithesis.bib}

% GB/T 7714-2015 is already enforced by tongjithesis.cls (biblatex style=gb7714-2015).
% Print DOI as plain text (no "DOI:" prefix, no hyperlink).
\DeclareFieldFormat{doi}{\nolinkurl{#1}}

\usepackage{duckuments}

\school
{土木工程学院}
{School of Civil Engineering}

\major
{土木工程}
{Civil Engineering}

\thesistitle
{基于雷视融合的塔吊吊装过程动态避障}
{与预警方法研究}
{Research on Dynamic Obstacle Avoidance and Early Warning Method for Tower Crane Hoisting Process Based on Radar-Vision Fusion}
{}

\thesisauthor{陈宇凡}{Yufan Chen}

\schoolnumber{2330894}

\keyword
{雷视融合,塔吊吊装,吊钩端感知,动态避障,风险预警}
{Radar-vision fusion, tower crane hoisting, hook-end perception, dynamic obstacle avoidance, risk early warning}

\advisor{卢昱杰教授}{Professor}

\thesisdate{2026}{2}{8}

\begin{document}

\maketitle

\begin{cabstract}
	塔式起重机(塔吊)承担了建筑施工中大量的垂直与水平运输任务,其作业过程具有载荷大、作业半径长、人员与设备交叉作业频繁等特点,安全风险长期处于高位。传统塔吊安全管理多依赖司机视野、指挥信号及固定式监控,难以对吊钩端附近的临近人员、临时障碍物以及动态变化的施工环境形成稳定、定量、可闭环的安全感知。

	针对上述问题,本文面向“吊钩端”这一关键危险源位置,构建双激光雷达与工业相机协同的轻量化感知系统,研究雷视融合条件下的空间定位、危险区域检测以及分级预警方法。首先,建立多传感器一体化安装与数据链路,完成相机内参、双雷达外参以及雷达—相机外参标定,并给出时间同步与延迟补偿策略;其次,提出双雷达点云实时融合与坐标统一方法,引入LIO-SAM实现吊钩端位姿估计与静态地图构建,在此基础上通过ROI裁剪、时空对齐和静动态分离实现目标生成与跟踪;最后,针对高风险动态目标(如作业工人)引入视觉侧识别结果,设计雷视一致性关联与置信度融合机制,构建基于距离、相对速度与时间到碰撞(TTC)的风险判定模型,实现静态障碍与动态目标的分级预警与策略输出。

	通过典型施工工况的对比实验,验证了所构建系统在定位精度、目标关联一致性以及实时性与稳定性方面的有效性,可为塔吊吊装过程的动态避障与风险预警提供工程化参考。
\end{cabstract}

\begin{eabstract}
	Tower cranes play a critical role in construction logistics, yet hoisting operations involve high loads and complex, dynamic environments, where workers and equipment often coexist within the working radius. Conventional safety approaches mainly rely on the operator's experience, manual signaling, or fixed surveillance, which lack reliable and quantitative perception around the hook end, leading to insufficient collision avoidance capability and delayed risk response.

	This thesis focuses on hook-end safety perception and proposes a dynamic obstacle avoidance and early warning method based on radar-vision fusion. A compact perception platform integrating dual LiDARs and an industrial camera is designed for hook-end deployment. A complete multi-sensor calibration pipeline is established, including camera intrinsics, LiDAR-LiDAR extrinsics, and LiDAR-camera extrinsics, together with time synchronization and delay compensation. For online perception, a real-time dual-LiDAR point cloud fusion method and unified coordinate transformation are developed. Hook-end localization and mapping are achieved using LIO-SAM, enabling motion compensation and consistent point cloud generation. On this basis, ROI cropping, spatio-temporal alignment, static-dynamic separation, and tracking are applied to produce reliable obstacle hypotheses. For high-risk dynamic objects (e.g., workers), vision-side detection results are fused with LiDAR observations via bidirectional consistency association and confidence fusion. A risk assessment model combining distance, relative velocity, and time-to-collision (TTC) is then proposed to drive a graded warning state machine.

	Experiments under representative hoisting scenarios demonstrate the effectiveness of the proposed system in localization performance, fusion consistency, and real-time stability, providing practical support for intelligent safety management in hoisting operations.
\end{eabstract}

% toc
\setcounter{tocdepth}{2} % show up to \subsection in ToC
\setcounter{secnumdepth}{2} % number up to \subsection
	oc

% main contents
\mainmatter
% Generate table of contents
\tableofcontents

\clearpage
\chapter{引~言}
\section{研究背景}

建筑业作为关系国计民生的基础性产业,是国民经济的重要组成部门。据国家统计局发布的《中国统计年鉴2023》数据显示,2022年建筑业总产值为83383.1亿元,占国内生产总值的6.9\%\cite{ref1}。在建筑施工作业中,塔式起重机是建筑施工现场中使用最广泛的垂直运输设备,具有起重高度高、作业范围广、工作效率高等优势\cite{ref2,ref3,ref4},其吊装作业承担着钢筋笼、模板、物料等关键构件的吊运任务。然而,塔吊施工环境普遍存在空间狭窄、视野受限、动态干扰因素复杂等特点,加之运行高度大、工作半径长,一旦发生碰撞或人员误入吊装路径,其后果往往极为严重。根据统计,2014年至2022年间全国起重机事故数量不低于220起;其中,吊物伤人与物体碰撞类事故累计超过75起,占比显著且发生频率较高\cite{ref5}。因此,如何实现吊装过程的实时环境感知与风险预警,是当前智慧工地安全管理中亟需攻克的重要问题之一。

近年来,随着建筑智能化与智能建造理念的提出,基于传感器的数据感知技术得到了快速发展,为施工过程数据采集的实时性和自动化需求提供了潜在的解决方案\cite{ref6}。大量研究集中于塔吊塔身或塔臂端的防碰撞系统,例如对多塔协同作业的塔臂干涉检测、塔机与周边建筑结构的距离监测等。然而,这类研究大多基于固定位置的传感器布置,主要关注塔臂或塔身层面的宏观防碰撞问题,而对实际吊装环节中最危险的部位塔吊吊钩端关注显著不足。

吊钩端是吊装过程中运动最频繁、最接近人员和构件的部位,其作业特点具有以下显著特征:(1)空间运动高度不确定:吊钩随吊臂回转、钢丝绳伸缩及外力扰动持续发生三维空间运动,摆幅可达数米,传统固定式传感器难以覆盖其动态空间范围。(2)近场作业风险极高:大量施工人员需要在吊运构件下方或附近进行配合作业,人员无意靠近吊钩、吊物摆动、构件碰撞均可能造成重大安全事故。(3)环境遮挡复杂且光照变化剧烈:钢筋笼、模板体系等大型构件会造成相机视野遮挡,而工地环境存在尘土、夜间照明等不利因素,对视觉系统效果影响显著。(4)传统方法难以实现实时、可落地的近场避障:现有塔吊安全系统普遍依赖塔身安装的超声、雷达或视觉设备,其检测范围难以覆盖吊钩周围 3\,\text{--}\,5 米的高危近场区域,使得关键危险点无法及时识别。

随着轻量化激光雷达和高性能嵌入式计算平台的快速发展,为塔吊吊钩端构建实时空间感知系统提供了技术基础。激光雷达具备高精度距离测量能力,能够在强光、弱光、粉尘等复杂工况中保持稳定输出,并可实现高达360°的水平视场(FoV),从而覆盖吊钩周围的全方位近场空间。相比之下,视觉传感器具有丰富的语义理解能力,特别是在识别施工人员、车辆等关键危险目标方面具有显著优势。因此,构建"雷达为主、视觉为辅"的雷视融合感知架构能够充分结合两类传感器的互补特性,为实现塔吊吊钩端的实时、可靠避障预警系统创造了可行条件。

基于上述背景,本研究面向吊钩端高风险作业场景,选用激光雷达与工业相机设计并开发集成式吊钩端感知系统,以实现对近场三维环境的高精度感知与危险目标识别,减少施工现场吊装作业中的碰撞风险、保障人员生命安全,也为建筑施工装备的数字化、智能化升级提供了一种可落地、可推广的技术路径,具有重要的工程应用价值和现实意义。


\section{研究目的与意义}

\subsection{研究目的}

针对塔吊吊装作业中吊钩端近场空间运动不确定、风险集中、环境干扰复杂等核心痛点,以及现有安全感知系统对吊钩端覆盖不足、单一传感器鲁棒性差、预警精度与实时性难以兼顾的问题,本研究旨在实现以下目标:
\begin{enumerate}
\item 设计并构建一套可工程化落地的吊钩端集成式感知系统,通过激光雷达与工业相机的协同部署,突破传统传感器在覆盖范围、环境适应性上的局限,实现对吊钩周围 3\,\text{--}\,5 米高危近场区域的全方位、高精度感知。

\item 提出基于雷视融合的多源数据处理与信息融合方法,整合激光雷达的三维几何测距优势与视觉传感器的语义识别能力,实现静态障碍精准建模、动态目标(尤其是施工人员)可靠识别与轨迹跟踪。

\item 建立兼顾空间距离与时间裕度的动态风险评估模型,形成"感知 - 识别 - 预判 - 预警"的闭环机制,实现对碰撞风险的分级预警与动态避障辅助决策,提升预警的提前量与准确性,降低误报、漏报率。

\item 验证系统在复杂施工工况下的稳定性、实时性与工程适用性,为塔吊吊装作业安全防护提供可推广、可复用的技术方案。
\end{enumerate}
\subsection{研究意义}

建筑业塔吊吊装事故频发,给人员生命安全与工程财产造成严重损失。本研究聚焦吊钩端这一最高风险区域,通过构建实时感知与预警系统,可直接减少人员误入、吊物碰撞等恶性事故发生概率,为施工现场安全管理提供技术保障。同时,系统采用轻量化硬件集成与高效算法设计,兼顾部署成本与运行稳定性,适配施工场地粉尘、震动、光照多变等恶劣环境,具备较强的工程落地性,可快速推广至各类建筑施工场景,推动塔吊装备的智能化升级,助力智慧工地建设。

现有塔吊安全感知研究多集中于塔身、塔臂等宏观层面,对吊钩端动态近场感知的研究较为匮乏。本研究提出 “雷达为主、视觉为辅” 的雷视融合架构,完善了多传感器在高危作业端的部署与标定方法,丰富了动态复杂场景下近场感知的理论体系。同时,研究建立的静动态目标分离、轨迹预测与分级预警模型,为多源数据融合在工程装备安全防护中的应用提供了新的思路,填补了吊钩端高精度、高鲁棒性感知技术的研究空白,为同类大型工程机械的安全智能化提供了理论参考与技术借鉴。

随着智能建造理念的深入推进,施工装备的数字化、智能化成为行业发展必然趋势。本研究通过传感器集成、数据融合与智能算法的深度结合,实现了吊装作业从 “被动防护” 向 “主动预警” 的转变,契合建筑行业安全升级与效率提升的核心需求。研究成果不仅可应用于塔吊设备,还可迁移至履带吊、汽车吊等其他起重机械,为工程机械行业的智能化转型提供典型示范,推动建筑业向安全、高效、智能的方向高质量发展。

\section{国内外研究现状}

随着智能建造与施工现场数字化水平的不断提升,工程机械与大型吊装设备的安全感知与智能防护逐渐成为国内外学术界和工程界的研究热点。围绕施工设备运行安全,已有研究主要集中在作业环境信息获取以及碰撞风险识别与预警等方面,形成了从感知、理解到决策的多层次研究体系。然而,由于施工现场环境复杂、目标类型多样、作业过程动态性强,现有研究在感知对象、传感器布置位置以及安全防护粒度等方面仍存在一定局限,尤其在塔吊等大型起重设备的吊钩端动态作业区域,相关研究尚不充分。
\subsection{吊装感知硬件布置方案研究}

吊装作业是施工现场最常见的高危作业之一。为防范作业过程中可能对人员或建筑构件造成的损害,众多学者致力于通过智能感知技术与传感器部署来提升作业安全性。例如,在塔吊吊臂、小车、驾驶室及施工现场周边等位置,布设摄像机\cite{ref7}、惯性测量单元(Inertial Measurement Unit,IMU)\cite{ref8,ref9}、射频识别(RFID)\cite{ref10}、超宽带(Ultra Wide Band,UWB)\cite{ref6}、超声波传感器\cite{ref11}、高度传感器\cite{ref12}、风速传感器\cite{ref13}等多种传感器,以实时监测作业状态、识别周边障碍物与人员,从而实现风险预警。

值得注意的是,传感器布设方案(含布设位置、传感器选型及集成方式)与传感精度直接决定环境感知效果,而感知数据作为后续风险识别、动态避障等算法判断的核心输入依据,其质量优劣直接影响决策的准确性与可靠性。不同布设位置、不同类型传感器所获取的数据,在感知范围、实时性、抗干扰性等方面存在显著差异,因此科学优化传感器布设方案,是提升吊装作业智能防护水平的关键前提。

\subsubsection{基于塔机结构本体的传感器布设方案}

在实际施工现场,多台塔吊交叉作业的场景普遍存在,作业空间交错复杂,极易引发塔吊间结构干涉、吊物与周边物体碰撞等叠加风险。早期塔吊防碰撞研究多依赖塔机自身运行参数进行间接判断,其中部分方案的传感器布设虽未直接聚焦塔臂,但为后续塔臂布设方案提供了基础思路。例如,周飞虎\cite{ref14}在塔机回转机构、变幅机构与起升机构等关键部位布设角度、幅度与高度传感器,采集塔机姿态与运行状态数据,通过几何关系推算吊钩位置及构件相对关系,将潜在碰撞转化为构件间几何距离与阈值的比较以触发预警。

进一步地,Zhong等\cite{ref15}从工程应用角度出发,构建了一套基于无线传感器网络(WSN)与物联网(IoT)的塔吊群安全管理系统。该系统在塔吊关键结构部位布设多类型传感器节点,包括塔臂、小车及起升机构等位置,用于采集塔吊运行状态与作业参数,并通过无线传感器网络实现多塔设备之间的数据传输与集中管理。传感器采集的数据主要包括塔臂回转角度、小车位置、吊钩高度等信息,系统基于这些参数在统一坐标系下对塔吊结构的空间关系进行计算,从而判断塔吊之间是否存在潜在的干涉或碰撞风险。当相关参数超过预设安全阈值时,系统通过物联网平台向操作人员发送预警信息,实现对塔吊群作业安全状态的实时监控。

针对塔吊间碰撞预警,Hwang等\cite{ref6}在实验室环境下搭建了两个缩尺塔吊模型,在每台塔吊吊臂前端和后端各布置1个UWB标签,并在实验场地角点固定布设UWB接收器,通过AoA与TDoA相结合的方式获取标签的空间位置信息。基于塔吊轴向旋转的运动特性,该方法采用标签间距离作为等效几何约束,将塔吊结构之间的安全距离映射为标签距离阈值,从而实现塔吊间潜在碰撞的实时预警。但该方案的传感器仍仅布设于塔吊结构本体,未在吊钩或吊物上增设独立感知单元,因此无法覆盖吊物与环境障碍物之间的碰撞风险,感知范围存在明显局限。

为直接感知吊臂周边障碍物,赵宇\cite{ref16}提出了一种基于超声波传感与无线组网的塔吊防碰撞方案,沿吊臂方向在吊臂关键位置均匀安装多个节点式超声波探测器阵列,用于实时感知吊臂周围障碍物的距离信息。由于单个超声波传感器视场角有限,该方法通过阵列化布设扩大探测范围,在吊臂周围形成连续的线性保护区域,可直接实现吊臂与障碍物之间的碰撞判断。同时,依靠部署在塔机回转机构和起升机构等关键部位的高度、角度传感器来获取吊物高度及塔身回转角度等参数,融合后推算出吊装物体的空间位置,从而间接实现吊装物体与障碍物之间的碰撞预警。

Zhou等\cite{ref17}将GNSS接收机固定于塔吊小车的支撑框架顶部,实时获取各塔吊塔臂端部在统一坐标系下的空间位置信息,从而实现对多塔空间关系的持续监测。基于GNSS解算得到的平面坐标数据,系统将塔臂简化为几何线段,并通过计算不同塔吊塔臂之间的相对距离来评估潜在的碰撞风险,当距离小于预设安全阈值时触发相应的碰撞预警。该方法通过引入GNSS定位传感器提升了多塔作业环境下的空间感知能力,但其传感器布设仍主要集中于塔吊结构本体,防碰撞对象以塔臂之间的结构干涉为主,未对吊钩或吊装物体进行直接感知。

张知田等\cite{ref18}提出面向"塔吊—工人"空间交互的危险场景自动检测框架,通过工人位置与吊钩(等效吊物)位置的同步获取实现预警。硬件布设上,该方法在塔吊驾驶室下方架设远视角摄像头以捕捉作业面内工人分布,同时在吊臂滑动小车下方安装垂直向下摄像头以估计吊钩/吊物的水平摆动偏移,并在塔吊主体布设角度与测距传感器获取吊臂旋转/下压角及小车、吊钩关键距离参数,通过局域网络实现数据实时传输。基于多源数据融合,方法先由图像估计吊钩偏移,再结合角度/测距数据计算吊钩三维坐标,并将工人与吊钩位置映射至BIM模型;在判别层面,以吊钩垂直投影点为中心构建10\,m圆形半径的动态危险区域,当工人进入该区域即触发危险场景预警。

上述方法中,传感器布设集中于塔机本体吊臂、小车等平台,工程部署难度低、成本可控,但因未直接感知周边环境,对周围障碍物及吊装作业的直接风险缺乏有效感知能力。这些方案大都通过传感数据组合联立,以计算出如吊钩、吊臂的特定位置,并与障碍物计算相对距离。然而,由于吊索与吊物的摆动、弹性变形以及吊钩端工作环境的复杂性,基于间接参数推算的定位方式往往存在累积误差与滞后问题,难以实现吊钩端近场的高精度、实时性感知。因此,部分工作进一步将传感设备直接部署在吊钩端/吊物,以更直接获取端部真实运动状态并提升预警可靠性。

\subsubsection{基于吊钩端的直接感知方案}

Lee等\cite{ref19}在塔吊盲吊导航系统中,将吊钩块作为关键观测对象:在变幅臂端安装激光传感器,并在吊钩块安装反射板作为测距靶标,利用激光束反射测得臂端到吊钩块的实际垂直距离。考虑到吊钩块剧烈摆动会导致"错失反射板"而产生误测,系统进一步结合吊钩块运动速度与方向对异常数据进行滤除,同时以放绳长度编码器作为备份测量源,最终将多源位置与视频信息发送至驾驶室内的导航服务器并在BIM模型中实时显示吊物位置与周边环境关系,主要用于提升盲吊场景下的态势感知与导航辅助。

在此基础上,为更直接表征"吊钩端摆动/旋转"等动态风险,Fang等\cite{ref20}提出将无线IMU刚性安装在吊物或吊钩端载荷上,由IMU测得吊物的欧拉旋转角,在已知吊绳长度并近似刚性的前提下,可将姿态角进一步转换为吊物在三维空间中的相对位置,从而实现对吊物摆动状态的在线监测,并与编码器等机构传感数据结合用于安全辅助与告警。

进一步地,Ku等\cite{ref21}在机器人化塔吊系统中将"吊钩块"视为末端执行器,直接在吊钩块安装3D LiDAR获取近场点云以感知周围结构与潜在障碍,并将点云处理流水线部署在吊钩块上的工业计算机中实时提取路径方向上的障碍高度等关键信息,处理结果再无线回传至驾驶室内主机用于决策与避障执行。

除此之外,由于摄像机轻便的特性,许多研究者也尝试将摄像机安置于吊钩端,对下方拍摄感知,实现如施工危险对象识别\cite{ref22}、工人与负载碰撞预警\cite{ref23}等塔吊安全监测与预防研究。


\subsection{施工现场环境感知方法研究}

施工现场环境感知是塔吊安全预警与避障控制的基础,其核心在于在复杂、动态且强不确定的工况下持续获取"设备自身状态—作业空间几何—人员与障碍物信息",并据此进行风险评估与提示。现有研究与工程应用中,塔吊相关感知传感器大体可按信息来源划分为三类:一类用于获取塔吊本体或吊钩端的运动状态与空间位姿,为后续感知与预警提供统一参考;第二类用于感知作业环境中的几何距离与空间结构,支撑安全距离判断与危险区域入侵检测;第三类用于获取目标的语义或身份信息,尤其是对工人等高风险对象的识别与管理。不同传感器在测量维度、精度、部署成本与环境鲁棒性方面存在显著差异,也决定了其在塔吊避障系统中常见的组合方式与融合层级选择。早期研究主要探索不同类型传感器在感知问题上的应用,随着研究不断深入以及感知判断要求的不断提升,多源传感器融合方式为提高系统鲁棒性、准确性提供了良好的解决办法。

\subsubsection{(1)传感技术}

编码器(包含角度、高度等状态传感器)常用于获取塔吊自身的状态,通过获取回转半径和回转角度等信息,能够推算出小车位置、吊钩高度以及结构关键部位的运动边界信息。该类传感器可靠性高、成本低、工程布置成熟,能够为塔吊运动学模型、作业包络计算与禁行区生成提供直接输入,是典型的"间接感知"手段。然而其测量对象主要是设备自身状态,难以感知外部环境中临时出现的障碍物、人员靠近以及吊物周边的局部复杂结构,因此通常需要与外部环境传感器结合才能形成完整的避障能力。

惯性测量单元(IMU)能够提供角速度与加速度信息,在吊钩端强动态条件下用于运动状态估计、姿态变化监测以及对扫描类传感器(如LiDAR)进行运动畸变补偿。对于吊钩端避障场景,IMU的价值主要体现在两方面:其一,用于描述吊钩摆动、旋转等快速姿态变化,为点云稳定化与坐标系一致性提供支撑;其二,与其他传感器(如LiDAR、视觉、GNSS)融合形成更稳定的位姿估计。但IMU存在累计漂移,单独使用难以提供长期稳定的位置与姿态基准,通常需要与外部定位或地图约束结合。

GNSS(及RTK)常用于塔吊场景中的宏观定位与设备状态监测,例如塔吊基座或关键构件的绝对位置获取、施工区域级的空间约束建立等。RTK具备较高的平面定位精度,适用于开阔环境下的绝对坐标基准构建,也便于与BIM/施工组织规划进行空间对齐。然而在城市密集施工环境中,GNSS易受遮挡与多径效应影响,且对吊钩端等局部高动态部位的精细定位支撑有限;此外,GNSS本身无法直接感知障碍物几何边界,因此更适合作为全局参考或与IMU、LiDAR等共同构建多传感器定位框架。

UWB定位系统在施工安全管理领域应用较多,典型方式是在工人佩戴标签、现场布置基站,从而实现人员定位、电子围栏与区域入侵告警。UWB的优势在于可对人员提供显式定位信息,便于实现"人"的风险管理与统计,并且可与塔吊作业区域规则进行结合。但UWB系统部署依赖基站布设与标定,定位精度与稳定性受环境遮挡、金属结构、多径效应影响明显;同时UWB难以描述障碍物的几何形状与边界,通常更适合作为人员高危提示或辅助决策信息。

RFID技术在塔吊作业中更多用于构件、吊具或物料的身份识别与流程管理,例如对吊装构件进行唯一标识、实现装卸与流转记录、辅助吊装任务调度与追溯。在安全预警层面,RFID可用于识别"正在吊装的对象是什么"、与作业权限或工序规则联动,从而在管理层面增强安全性。但RFID通常难以提供稳定连续的空间位置与距离信息,其通信距离、读写可靠性也受金属遮挡与现场电磁环境影响,因此更适合作为"身份/状态信息源"融入系统。

视觉作为一种信息丰富且价格便宜的信息获取方式,在语义理解方面具有明显优势,能够直观区分不同目标类别。早期研究者就已尝试在塔吊上部署视觉系统以显示吊装的负载和下方的工作区域,让操作员能够直观看到吊钩端的情况,减少视觉盲区。如今,通过结合深度学习算法,能够实现施工对象的目标检测,进而实现吊物空间姿态测量、生产效率统计、工人运动轨迹监测等任务。然而,视觉感知方法对光照条件和环境遮挡较为敏感,在夜间作业、强光或复杂遮挡条件下感知性能容易下降,同时难以直接获取精确的空间距离信息。深度相机与结构光相机能够直接提供局部深度信息,适合近距离空间测量与区域入侵检测,但在强光环境、远距离以及室外复杂工况下稳定性和测距范围受到限制。

毫米波雷达具备全天候工作能力,对雨雾粉尘等恶劣工况鲁棒,且对运动目标具有较强敏感性。其不足在于角分辨率与空间细节表达能力通常不如LiDAR,难以精确刻画复杂障碍物的几何边界,因此更常用于冗余检测、近距离存在性判断或对运动目标的补充提示。

超声波传感器成本低、实现简单,适合极近距离防撞与边界触发。但其测距范围有限,受传播介质与风噪影响较大,难以覆盖吊钩端大范围三维空间,因此更多作为近距冗余或辅助触发传感器。

激光雷达(LiDAR)能够输出三维点云并提供稳定的测距能力,是施工现场空间几何感知的关键传感器之一。对于塔吊避障而言,LiDAR可直接支持危险区域入侵判定、安全距离阈值触发、障碍物轮廓提取等任务,并且对光照变化不敏感,具备较高工程鲁棒性。但LiDAR对语义类别理解较弱,对"人"与"非人"的区分通常需要结合视觉或学习型算法;此外,金属反射、玻璃等材质以及遮挡引起的点云缺失需要配合滤波、聚类与一致性策略提高稳定性。因此在实际系统中,LiDAR常作为"测距与空间判定主传感器"与其他语义或状态传感器形成互补。

\subsubsection{(2)多传感器融合感知}

在复杂施工现场,仅依赖单一传感器往往难以同时满足安全预警所需的空间几何精度、语义理解能力与环境鲁棒性。因此,多传感器融合逐渐成为施工环境感知的重要技术路线。一般而言,多传感器融合是指在统一的时间与空间参考框架下,对来自不同传感器的信息进行对齐、关联与组合,以获得更完整、更可靠的环境状态估计或风险判定结果。其本质目标可概括为两类:其一,通过不同模态信息的互补性提升感知维度;其二,通过信息的冗余性增强系统可靠性与容错能力。

在塔吊避障与安全管理领域,多传感器融合研究大体可以分为两类路径:一类面向塔吊/吊钩端状态与位姿获取,以编码器、限位、GNSS/RTK、IMU等为主要信息源,通过融合提升位姿连续性与全局参考稳定性,为作业包络计算与塔群协同防碰撞提供支撑。许多研究者采用编码器与倾角传感器测量吊臂回转角、提升绳伸长量和吊臂伸长量,根据各部分参数通过建立塔吊的运动方程得到塔吊自身的状态。为直接感知吊钩摆动,有研究在吊钩上侧安装IMU捕捉其运动姿态,也有采用视觉系统来定位吊物的偏移和旋转,但缺乏吊物的深度信息。

另一类路径面向外部环境与人员风险识别,更强调对障碍物几何距离与人员语义信息的联合获取。典型做法包括以激光雷达/毫米波雷达提供可靠测距与空间入侵判定,以视觉方法提供工人等高危对象的语义识别与可视化解释。在实际研究中,视觉传感器因能同步获取障碍物几何距离与人员语义信息的独特优势,成为多传感器融合系统的核心感知单元。Yang等采用Mask R-CNN对人员与障碍物进行语义分割,通过像素-实际距离转换模型,联合输出人员语义状态与安全距离;Golcarenarenji等设计的CraneNet深度学习模型,基于单目视觉实现50米范围内人员检测与远距离测量,准确率达92.59\%。双目视觉方案则通过立体匹配直接获取障碍物距离,几何精度高于单目,但其测量范围受基线长度限制。

视觉与其他传感器的融合方案进一步提升了联合感知的鲁棒性与精度。视觉与激光雷达融合方案中,视觉负责人员、设备、障碍物的语义识别,激光雷达提供高精度几何距离与三维点云,通过特征级融合将距离特征融入视觉检测网络,弥补视觉测距精度不足与激光雷达语义信息匮乏的缺陷。视觉+RFID/IMU融合方案则利用RFID/IMU的精准定位能力与视觉的行为语义提取优势,通过决策级融合实现人员动态风险预警。

\subsubsection{(3)雷视融合感知}

在智能交通、自动驾驶、智慧工地等复杂环境感知场景中,雷达与视觉传感器作为核心感知载体各有优劣且高度互补。视觉传感器语义信息提取能力强、成本低、部署灵活,可精准识别目标类别与外观特征,但易受光照、天气及遮挡影响;激光雷达能输出精准三维点云,毫米波雷达适应恶劣天气且动态跟踪效果好,但语义信息匮乏。雷视融合技术通过数据级、特征级等合理融合策略整合二者数据,借助视觉语义优势弥补雷达短板,同时依托雷达空间感知能力修正视觉偏差,显著提升复杂场景适应能力与感知鲁棒性。

从融合策略与信息处理层级出发,现有研究将雷视融合划分为数据级融合、特征级融合、目标属性级融合与决策级融合四类。

数据级融合强调在原始数据层面实现信息联合,对多源测距数据进行坐标统一与点云拼接、对图像与深度/点云进行时空对齐。Zhou和Omar采用像素级融合红外图像与RGB图像,针对饱和像素区域实施主成分分析算法。Kim等综合使用GPS、毫米波雷达、LiDAR与视觉构建环境表示,通过多帧LiDAR观测累积生成网格地图。Lekic与Babic利用GAN将雷达信息生成环境图像并与光学图像融合。

特征级融合通常先从各传感器数据中提取几何特征、运动特征或深度网络特征的中间表示,再在特征空间完成联合建模。在深度学习框架中,特征融合通常被嵌入神经网络结构中,以端到端方式自动学习跨模态互补表示。Caltagirone等基于全卷积网络实现LiDAR与相机融合的道路检测。

更贴近工程避障预警需求的是目标属性级融合,各传感器先输出目标候选,再进行融合以降低单一传感器误报与漏报。一个与"吊钩端避障可视化"高度一致的范式是"LiDAR生成ROI→图像中识别/解释ROI"。Wu等利用LiDAR的距离与角度信息在图像中生成ROI,并结合3D点云形状信息进一步验证目标以降低误报。Han等从系统实现角度强调实时性与工程落地。Zhong等利用毫米波雷达运动信息提供图像ROI,并在ROI上用CNN识别目标。

决策级融合由各传感器独立完成检测、识别或风险判定,再通过规则、置信度或逻辑推理进行融合输出,如采用贝叶斯推理、D-S证据理论、模糊推理等。Wei等直接面向工业场景的实时避碰系统,体现了"检测结果融合→避障决策"的应用导向。

综上所述,雷视融合研究的核心共识在于:通过融合激光雷达的高精度三维几何与测距能力以及相机的纹理与语义表达能力,可以在复杂动态环境中显著提升目标感知的完整性、鲁棒性与可解释性。现有工作普遍强调"可靠的时空对齐与外参标定"是雷视融合有效性的基础。面向安全预警与避障等工程任务,雷视融合的研究趋势更强调在实时性、系统复杂度与安全可靠性之间取得平衡:利用几何信息实现稳定的距离与区域风险触发,同时用视觉语义增强对高风险目标的识别与可视化解释,从而构建"可落地、可维护、可扩展"的感知—预警闭环。

\subsection{碰撞检测算法与预警模型研究}

关于塔吊碰撞的检测与约束建模,既有研究首先在塔吊吊装规划与虚拟仿真场景中形成了较为系统的方法体系\cite{ref45}。在该场景下,研究人员多采用虚拟环境构建手段以模拟实际吊装场景,并以碰撞检测作为4D路径规划的约束条件\cite{ref46},如通过为塔吊构建定向包围盒(Oriented bounding box,OBB)\cite{ref47},采用离散碰撞检测(Discrete Collision Detection,DCD)\cite{ref48,ref49}或最小距离计算\cite{ref50}等方式判断碰撞是否发生。Lai和Kang\cite{ref51}中以简化外边界体(如球体、圆柱体等)近似现场机械与结构构件,从而显著降低实时碰撞检测的计算负担。这些方法在虚拟仿真环境下具有可行性,因为此时障碍物信息通常是已知且假设完全可观测的。但在实际场景中,周围环境的信息需要通过传感器实时获取,传感器的观测范围、精度以及数据延迟等问题,单纯依赖离线/静态几何模型的碰撞约束往往难以直接迁移。

针对施工过程中的实时安全辅助,研究逐渐从"是否碰撞"的二值判别,转向"碰撞风险强度"与"分级预警"建模。一类典型思路是构建安全包络或风险区域,采用最小距离/侵入判别触发告警\cite{ref52}。Wang等\cite{ref53}将风险区域与目标简化为AABB,并通过进入/逼近判别实现安全预警逻辑,体现了"几何包络 + 阈值规则"在实时系统中的工程可实现性。Yang等\cite{ref60}使用Mask R-CNN检测危险源并根据照片与实际的像素转换关系换算计算工人与危险源的安全距离。

更进一步的研究强调在作业过程中持续评估风险并向操作者提供主动式辅助,面向吊装作业的实时主动安全辅助框架的核心在于动态工作空间建模与风险提示机制,而非仅在碰撞发生时被动报警\cite{ref20}。Fang等\cite{ref9}进一步从人机协同角度讨论了安全风险呈现与操作者态势感知评估,为预警模型如何"可解释地"服务现场作业提供了参考。

当障碍物与塔吊/吊载存在相对运动时,仅以"当前距离是否小于阈值"进行判断可能出现两类问题:其一,两个目标距离尚大但存在快速相向趋势,风险会在短时间内迅速上升;其二,阈值设置过大又会带来高误报,影响作业效率。因此,很多研究引入时间域代理风险指标,将风险刻画从"空间距离"推进到"时间余量",典型如Time-to-Collision(TTC)等。虽然TTC最初广泛用于交通安全代理指标的研究语境,Nadimi等\cite{ref54}对TTC等指标的适用性与局限进行了系统讨论,为"时间裕度型预警阈值如何设定与解释"提供了可借鉴的建模思路。在吊装场景下,将TTC类思想与相对速度/加速度估计结合,可形成"距离—速度联合预警"的框架,使预警更关注"是否会在未来短时窗内进入危险区",而非仅关注当前几何接触。

根据Kim等\cite{ref61}的研究,相较于预防邻近的事故,预测潜在的碰撞事故能够更为有效地避免危险。围绕预测驱动的碰撞预警,研究通常先对作业对象进行短时轨迹预测,再对未来时窗内的碰撞可能性进行判别或分级。张冬\cite{ref62}将碰撞预警方式区分为两种形式:针对塔吊与静止建筑物的碰撞,考虑为基于距离的预警;而针对塔吊与塔吊之间的碰撞,在距离模型的基础上结合速度模型来进行综合判断。进一步地,轨迹预测方法本身也形成了从统计模型到深度学习模型的谱系,例如通过隐马尔可夫模型(Hidden Markov Model,HMM)\cite{ref63}、长短期记忆人工神经网络(Long Short-Term Memory,LSTM)\cite{ref64}、深度神经网络(Deep Neural Network,DNN)\cite{ref65}等对人员或障碍物运动进行建模与预测。在获得塔吊与障碍物的未来轨迹后,可通过综合分析双方的相对运动趋势来进行风险判断,例如采用基于碰撞锥\cite{ref66,ref67}、基于约束优化等方法\cite{ref68}。

除风险度量模型外,动态场景下的碰撞检测还面临"实时性—精度/漏检风险"的工程折中问题。为此,相关研究在算法实现上通常采用"分层剔除 + 精细检测"的策略。在吊装路径规划与动态环境中,Dutta等\cite{ref48}提出了近实时重规划模块,并在决策支持部分采用多层级OBB来进行风险相关的判别与触发,体现了"层次包围体 + 决策逻辑"在动态场景中的可扩展性。Zhu等\cite{ref55}强调为提高碰撞检测效率与精度,需要融合空间划分与层次包围体等策略,并讨论了连续碰撞检测(CCD)在吊装扫掠体场景下的重要性。同样地,Lin等\cite{ref56}提出基于点云的碰撞检测方法并将其嵌入规划流程,反映出"直接用现场环境状态驱动碰撞判别"的趋势。

在更一般的动态避碰理论中,碰撞风险还可通过"相对速度可达集合"进行刻画,例如碰撞锥(collision cone)与速度障碍(velocity obstacle, VO)范式。Fiorini和Shiller早在1998年便提出了VO框架以在动态环境中选择规避机动\cite{ref57},其思想可用于将塔吊端执行体与动态障碍的相对运动关系显式化。Van den Berg等\cite{ref58}将该思想扩展到多主体实时避碰,为复杂多障碍、多作业体并行施工条件下的风险判别提供了更系统的理论工具。进一步地,安全关键建模常将避碰问题表述为"满足安全集约束"的控制问题,控制障碍函数(CBF)成为近年来重要方向之一。Jian等\cite{ref59}提出了D-CBF,将障碍物预测与安全约束结合,用以保证动态避障的可行性与安全性,这为"预警模型如何与后续控制/干预策略耦合"提供了可迁移的建模范式。

综上,塔吊碰撞检测与预警模型研究呈现出从"仿真/规划约束下的几何碰撞判别"向"面向施工过程的分级预警"再到"时间裕度与预测驱动风险评估"的演进趋势。一方面,距离/包络等几何判别仍是预警触发的基础,但在动态场景下需结合时间裕度指标与轨迹预测以提升提前量并降低误报;另一方面,为实现在线运行,碰撞检测算法本身也趋向采用层次包围体、空间划分与连续碰撞检测等策略来兼顾效率与漏检风险。上述研究为后续构建面向塔吊作业的实时碰撞预警系统提供了可复用的方法基础,同时也提示预警模型需要在提前量、误报/漏报、实时性与鲁棒性之间进行系统权衡。

\subsection{文献总结}

综合国内外研究现状,围绕塔吊吊装安全的感知硬件布置、环境感知方法、碰撞检测与预警模型三大核心方向,现有研究已形成一定的技术积累,但仍存在以下关键不足,为本文研究提供了明确的切入点:

\begin{enumerate}
\item \textbf{感知硬件部署聚焦不足:}现有传感器布置多集中于塔吊塔身、塔臂或施工现场固定位置,以宏观防碰撞为目标,对吊钩端这一最高风险区域的直接感知研究匮乏。虽有少数研究尝试在吊钩端部署传感设备,但存在硬件维护困难、抗干扰能力弱、感知范围有限等问题,难以满足近场全方位、高精度感知需求。

\item \textbf{环境感知方法存在局限:}单一传感器感知存在明显短板(如视觉受光照影响、雷达语义匮乏),多传感器融合成为主流趋势,但现有融合方案或过于依赖复杂深度学习模型导致实时性不足,或仅停留在数据拼接层面未能充分发挥互补优势。针对施工场景的雷视融合研究,尚未形成兼顾几何精度、语义识别与工程鲁棒性的成熟方案,尤其缺乏针对吊钩端动态特性的适配优化。

\item \textbf{预警模型适配性不足:}现有碰撞检测算法多源于虚拟仿真或路径规划场景,依赖静态几何模型,难以应对施工现场动态干扰多、障碍物复杂的实际情况;预警机制多基于单一空间距离阈值,未充分结合目标运动趋势与时间裕度,易出现误报或预警滞后问题。同时,对施工人员等高风险目标的语义关联不足,难以实现精准分级预警。
\end{enumerate}
综上,吊钩端感知系统与雷视融合方法为塔吊安全提供了新的技术路径,但仍面临:吊钩端部署条件受限、传感器标定与同步困难、融合关联易受遮挡与稀疏性影响、预警策略缺少与工程动作相匹配的分级机制等问题。本文将围绕系统构建、融合定位、危险检测与分级预警开展研究,以形成可验证、可集成的完整方案。

\section{主要研究内容}
结合工程需求与技术难点,本文的主要研究内容包括:
\begin{enumerate}
\item 吊钩端多传感器系统构建:完成双激光雷达与工业相机的选型、布局优化与一体化盒体设计,并建立数据链路。
\item 基于融合点云的定位与危险检测:实现双雷达点云实时融合与发布,基于LIO-SAM完成位姿估计与静态地图构建,进而进行ROI裁剪、静动态分离与目标跟踪,实现空间危险区域检测。
\item 基于雷视融合的避障预警策略:构建视觉侧高风险目标识别接口,设计双向一致性关联与置信度融合,建立风险判定模型与分级预警状态机。
\end{enumerate}

\section{技术路线}
本文的技术路线如图\ref{fig:tech_route}所示:以吊钩端多传感器系统为数据入口,通过标定与同步实现统一时空基准;在点云侧完成融合与定位建图,在视觉侧完成高风险目标识别;最终通过雷视融合、风险评估与状态机输出预警与策略。

\begin{figure}
	\centering
	\includegraphics[width=.85\textwidth]{example-image-duck}
	\caption{技术路线示意图}
	\label{fig:tech_route}
\end{figure}

\chapter{吊钩端传感器系统构建与配准}
\section{系统总体架构与硬件选型布置}
\subsection{传感设备选型}
吊钩端感知系统需要在有限空间内实现对周边危险区域的覆盖,并满足轻量化、抗振动、防尘防水与快速安装维护等要求。本文采用“双LiDAR + 工业相机”的异构配置:双LiDAR以不同俯仰角与安装基线覆盖近场盲区并提升点云密度;工业相机面向作业区域提供语义识别能力。系统采用一体化盒体设计,将传感器、计算单元与供电/通信模块集成于吊钩端安装结构中,减少外露线缆并提升可靠性。

\begin{figure}
	\centering
	\includegraphics[width=.6\textwidth]{example-image-duck}
	\caption{吊钩端多传感器一体化安装示意图(示意)}
	\label{fig:hook_mount}
\end{figure}

\begin{table}
	\centering
	\caption{吊钩端传感器系统的选型与关键指标(示例)}
	\begin{tabular}{lccc}
		oprule
	组件 & 作用 & 关键指标 & 备注\\
	\midrule
	LiDAR-1 & 近场几何感知 & 360$^{\circ}$水平视场、$\ge$10Hz & 倾角覆盖吊钩下方\\
	LiDAR-2 & 远/侧向补盲 & $\ge$10Hz、同步触发 & 与LiDAR-1形成基线\\
	工业相机 & 语义识别 & 全局快门、$\ge$25fps & 支持低照度\\
	计算单元 & 在线处理 & 多线程、硬件加速 & 边缘侧部署\\
	\bottomrule
	\end{tabular}
	\label{tab:sensor_spec}
\end{table}

\subsection{安装布局优化}
(本节待补充。)

\subsection{吊钩端一体化盒体设计}
(本节待补充。)

\section{多传感器外参标定}
为保证多源数据在同一坐标系下可融合处理,需要完成以下标定流程:
\begin{enumerate}
\item 相机内参标定:获取焦距、主点与畸变参数,用于图像去畸变与后续投影。
\item 双雷达外参标定:求取两雷达坐标系之间的刚体变换,实现点云统一。
\item 雷达—相机外参标定:建立点云到图像的投影关系,为跨模态关联与一致性验证提供基础。
\end{enumerate}

一般地,坐标变换可表示为齐次矩阵形式:
\begin{equation}
\mathbf{T}_{a}^{b}=\begin{bmatrix}
\mathbf{R}_{a}^{b} & \mathbf{t}_{a}^{b}\\
\mathbf{0}^\mathsf{T} & 1
\end{bmatrix},\qquad
\mathbf{p}_{b}=\mathbf{T}_{a}^{b}\,\mathbf{p}_{a}
\end{equation}
其中,$\mathbf{R}_{a}^{b}$为旋转矩阵,$\mathbf{t}_{a}^{b}$为平移向量。

\section{数据链路与接口规范}
系统的数据链路包括传感器数据采集、时间同步、在线处理与结果发布。为降低多传感器异步带来的融合误差,本文采用时间戳对齐与延迟补偿策略:对各传感器采样时间进行统一基准标定,在融合侧按统一时间轴进行插值或最近邻匹配。接口规范方面,定义统一坐标系(吊钩端坐标系、雷达坐标系、相机坐标系与世界坐标系)以及TF发布关系,规定点云、检测框、跟踪目标与预警状态等消息字段,便于模块化集成与工程扩展。

\chapter{基于融合点云的定位与空间危险区域检测方法}
\section{双雷达点云实时融合方法}
双雷达点云融合的核心在于:基于外参将两路点云变换至同一参考坐标系,对时间差进行补偿,并在空间上进行去重与密度一致化处理。本文以吊钩端坐标系为参考,将LiDAR-2点云通过$\mathbf{T}_{2}^{1}$变换到LiDAR-1坐标系下,并进行体素滤波与地面/吊钩结构点剔除,输出融合点云用于后续定位与检测。

\section{基于融合点云的SLAM定位与静态地图构建(LIO-SAM)}
吊钩端平台存在摆动与运动变化,点云会产生运动畸变。本文选用LIO-SAM框架进行位姿估计与建图:利用点云特征提取与因子图优化得到连续位姿输出,同时结合IMU预积分实现更稳定的短时运动约束。在位姿输出的基础上,对点云进行去畸变处理并累积生成静态地图,为危险区域检测提供环境先验。

\section{面向目标提取的点云标准化与静动态目标生成}
为适应施工现场动态变化,本文采用“ROI裁剪—时空对齐—静动态分离—目标跟踪”的流程生成可用目标:首先基于吊钩端工作半径与高度范围对融合点云进行ROI裁剪;其次利用位姿估计结果完成时空对齐,将连续帧点云变换到同一参考系;然后通过与静态地图差分或基于聚类的前景提取实现静动态分离;最后对动态目标进行多目标跟踪,输出目标的状态(位置、速度与置信度)。

\begin{algorithm}
	\caption{双雷达点云融合与危险区域检测流程(示例)}
	\label{algo:fusion_risk}
	\KwIn{两路点云$\mathcal{P}_1(t)$、$\mathcal{P}_2(t)$,外参$\mathbf{T}_{2}^{1}$,位姿$\mathbf{T}_{w}^{h}(t)$,ROI范围$\Omega$}
	\KwOut{融合点云$\mathcal{P}_f(t)$,危险目标集合$\mathcal{O}(t)$,风险等级$L(t)$}
	对$\mathcal{P}_1(t)$、$\mathcal{P}_2(t)$进行时间对齐与去畸变\;
	将$\mathcal{P}_2(t)$变换到雷达1坐标系:$\mathcal{P}_2'\leftarrow \mathbf{T}_{2}^{1}\,\mathcal{P}_2(t)$\;
	融合并滤波:$\mathcal{P}_f(t)\leftarrow \text{VoxelFilter}(\mathcal{P}_1(t)\cup \mathcal{P}_2')$\;
	ROI裁剪:$\mathcal{P}_\Omega\leftarrow \text{Crop}(\mathcal{P}_f(t),\Omega)$\;
	静动态分离:$\mathcal{P}_{dyn}\leftarrow \text{ForegroundExtract}(\mathcal{P}_\Omega,\text{Map})$\;
	聚类与跟踪:$\mathcal{O}(t)\leftarrow \text{Track}(\text{Cluster}(\mathcal{P}_{dyn}))$\;
	风险评估与分级:$L(t)\leftarrow \text{RiskStateMachine}(\mathcal{O}(t))$\;
	\Return{$\mathcal{P}_f(t),\mathcal{O}(t),L(t)$}
\end{algorithm}

为实现定量化风险评估,可构建包含距离与相对运动的风险评分:
\begin{equation}
R(t)=w_d\cdot \frac{1}{d(t)+\epsilon}+w_v\cdot \max\bigl(0, v_{rel}(t)\bigr)+w_{ttc}\cdot \frac{1}{\text{TTC}(t)+\epsilon}
\end{equation}
其中,$d(t)$为目标到吊钩端的最小距离,$v_{rel}(t)$为沿连线方向的相对速度,$\epsilon$为数值稳定项。

\begin{proposition}
\label{prop:ttc}
设目标与吊钩端的最小距离为$d(t)$,相对接近速度为$v_{rel}(t)>0$,且短时间内$v_{rel}(t)$近似恒定,则时间到碰撞满足$\text{TTC}(t)=\frac{d(t)}{v_{rel}(t)}$。当$\text{TTC}(t)\le \tau$时,存在在阈值时间$\tau$内发生接触风险的充分条件。
\end{proposition}

\begin{proof}
当$v_{rel}(t)>0$且近似恒定时,距离随时间变化满足$\dot d(t)=-v_{rel}(t)$,积分得$d(t+\Delta t)=d(t)-v_{rel}(t)\,\Delta t$。令$d(t+\Delta t)=0$可得$\Delta t=\frac{d(t)}{v_{rel}(t)}$,即$\text{TTC}(t)$表达式成立。当$\text{TTC}(t)\le\tau$时,取$\Delta t=\text{TTC}(t)$则$\Delta t\le\tau$,在$\tau$内可达到$d=0$,因此构成接触风险的充分条件。
\end{proof}

\chapter{基于雷视融合的避障预警策略}
\section{视觉侧高风险目标识别}
施工现场中,作业人员是最典型的高风险动态目标。本文在视觉侧采用工人检测模型输出目标类别与2D框,并结合时序稳定化处理(例如基于跟踪的短时一致性约束)降低漏检与抖动。视觉侧输出通过统一接口发布,包括时间戳、检测置信度、2D框与(可选)关键点信息,为后续雷视融合关联提供语义先验。

\section{雷视融合策略}
雷视融合采用“几何约束 + 语义一致性”的策略:首先将跟踪得到的3D目标在外参关系下投影到图像平面,形成候选投影区域;其次与视觉2D检测框进行双向一致性匹配(点云投影落入2D框比例、2D框内投影点密度等);最后完成轨迹级绑定,将视觉语义标签绑定到点云目标轨迹上,并采用置信度融合获得更稳定的目标存在概率。

\begin{figure}
	\centering
	\includegraphics[width=.8\textwidth]{example-image-duck}
	\caption{雷视融合关联示意图(示意)}
	\label{fig:rv_association}
\end{figure}

\section{风险判定与分级预警策略}
预警策略需要兼顾安全性与可用性。本文将风险来源分为静态障碍与动态目标两类:
\begin{enumerate}
\item 静态障碍预警:基于静态地图与当前位姿,判断吊钩端运动方向上的安全包络是否与静态结构发生交叠,并结合距离阈值触发预警。
\item 动态目标风险评估:对雷视融合后的高风险目标,计算$d(t)$、$v_{rel}(t)$与$\text{TTC}(t)$等指标,得到风险评分$R(t)$。
\item 分级预警状态机:按照“提示—警告—危险”三级输出,设置进入/退出滞回与最小持续时间,避免频繁跳变,并输出建议动作(减速、暂停、人工确认)。
\end{enumerate}

\begin{lstlisting}[language=c++,caption=风险评估与分级预警的核心计算(示例),escapechar=@]
// Inputs: distance d (m), relative approaching speed v@\_@rel (m/s)
// Output: risk level (0: safe, 1: notice, 2: warning, 3: danger)
double eps = 1e-3;
double ttc = (v@\_@rel > 0.0) ? (d / (v@\_@rel + eps)) : 1e9;
double R = wd / (d + eps) + wv * std::max(0.0, v@\_@rel) + wttc / (ttc + eps);

int level = 0;
if (ttc < tau@\_@danger || R > R@\_@danger) level = 3;
else if (ttc < tau@\_@warn || R > R@\_@warn) level = 2;
else if (d < d@\_@notice) level = 1;
else level = 0;
\end{lstlisting}

\chapter{实验验证与系统评估}
\section{实验平台与测试场景设置}
实验平台由吊钩端多传感器一体化盒体、边缘计算单元与上位机监控组成。测试场景覆盖典型吊装工况:静态障碍(脚手架、堆料区边界等)临近、人员从不同方向接近、吊钩端摆动与路径变化等。

\section{多传感器定位性能实验}
定位实验主要验证位姿精度与稳定性:通过对比基准测量(例如全站仪/高精度定位设备或离线精标结果)评估平移与航向误差;同时在摆动工况下分析位姿估计的短时波动与漂移特性,验证去畸变对点云一致性的提升。

\section{雷视协同感知效果分析}
视觉侧评估关注工人检测的准确率与时序稳定性;融合侧评估关注跨模态关联一致性(正确绑定率、误绑定率)以及在遮挡/稀疏点云条件下的鲁棒性。通过对比“仅视觉”“仅点云”“雷视融合”三种模式,分析融合策略对漏检与误报的影响。

\section{危险区域检测效果验证}
危险区域检测采用静态障碍与动态人员接近两类测试。静态障碍测试验证安全包络与障碍边界的检测准确性;动态人员接近测试验证TTC与风险评分驱动的预警触发提前量与稳定性,重点关注预警对实际工况的可解释性与可操作性。

\section{系统实时性与稳定性验证}
从工程落地角度,本文对各模块耗时进行统计(点云融合、定位、检测/跟踪、雷视关联、风险评估与发布),验证在目标刷新频率与计算资源约束下的实时性;同时通过长时间运行测试观察数据链路丢包、时间同步漂移与系统恢复能力,评估工程可行性。

\begin{longtable}{l|p{9cm}}
\caption {系统模块与功能说明(示例)}\label{Tab:modules}\\
	oprule
	extbf{模块} & \textbf{功能描述}\\\midrule
\endfirsthead
\captionsetup{style=continued}
\caption[]{}\\
\midrule
{\bf 模块} & {\bf 功能描述}\\\midrule
\endhead
\endfoot
\endlastfoot
数据采集与同步 & 采集双LiDAR点云与相机图像,完成时间戳对齐与延迟补偿,输出统一时间基准的数据流。\\
外参管理与TF发布 & 管理相机/雷达外参,发布吊钩端坐标系与各传感器坐标系之间的TF关系,支持在线校验。\\
点云融合与预处理 & 进行点云变换、体素滤波、ROI裁剪与结构点剔除,输出融合点云用于定位与检测。\\
定位建图(LIO-SAM) & 基于点云特征与IMU约束估计位姿,进行去畸变与静态地图构建,为危险检测提供环境先验。\\
目标生成与跟踪 & 对动态前景点云聚类并跟踪,输出目标位置、速度与轨迹置信度。\\
视觉检测与时序稳定 & 识别工人等高风险目标并做时序平滑,输出2D框与置信度,提供融合接口。\\
雷视关联与置信度融合 & 将3D目标投影到图像,进行双向一致性匹配,完成轨迹级绑定与置信度融合。\\
风险评估与预警输出 & 基于距离、相对速度与TTC计算风险评分并驱动分级状态机,发布预警等级与建议动作。\\
实验与评估工具 & 记录日志、回放数据、统计模块耗时与评估指标,用于性能验证与参数调优。\\
\bottomrule
\end{longtable}

\chapter{结论与展望}
\section{主要研究结论}
本文面向塔吊吊装过程的吊钩端安全风险,构建了双LiDAR与工业相机融合的吊钩端感知系统,完成多传感器标定与同步;提出了融合点云的实时生成、基于LIO-SAM的定位建图以及面向危险区域检测的目标生成与跟踪流程;设计了雷视一致性关联与置信度融合策略,并建立基于距离、相对速度与TTC的分级预警方法。实验结果表明,所提出方法能够在典型工况下实现稳定的危险目标感知与及时预警输出。

\section{创新点总结}
本文的主要创新点可概括为:\,(1) 面向吊钩端部署的双LiDAR+相机一体化感知系统与接口规范;\,(2) 融合点云驱动的吊钩端定位与危险目标生成流程,适配摆动与动态环境;\,(3) 面向高风险目标的雷视双向一致性关联与置信度融合,以及可工程落地的分级预警状态机。

\section{存在的不足与未来展望}
本文仍存在以下不足:吊钩端平台的强振动与遮挡工况对感知稳定性仍有影响;视觉模型在极端光照与严重遮挡条件下性能下降;风险阈值与状态机参数具有一定场景依赖性。未来可从轻量化多模态网络、端到端时空融合、基于行为预测的风险评估以及与塔吊控制系统的闭环联动等方向进一步研究。



\clearpage
{
	\hyphenpenalty=1000 %断词阈值,值越大越不容易出现断词
	\tolerance=500 %丑度,10000为最大无溢出盒子
	\hbadness=100 %如果丑度超过hbadness这一阀值,那么就会发出警告
	\printbibliography[heading=bibintoc,title=参考文献]
}

\clearpage
\addcontentsline{toc}{chapter}{本科就读期间取得的学术成果}
\chapter*{本科就读期间取得的学术成果}
\begin{enumerate}
	\item 1
	\item 2
	\item 3
	\item 4
\end{enumerate}

\clearpage
\addcontentsline{toc}{chapter}{致~谢}
\chapter*{致~谢}

我衷心感谢\textsc{TongjiThesis}对我的论文的帮助,帮助我节约了不少时间。非常感谢!

\begin{enumerate}
	\item 股女济增型馆发程情少我,初答积老西,去时们不要,小的而自回念因的政根我节本史。
	\item 以里生星故神始广意金施头、儿于洋小,方根心起热外一轮了以定只有到本治:像士己不无回念年洲,心格打导解美,国一西的己一场?
	\item 一管牛里向的,一到的这出脑竟送。
	\item 如感百是候,一心台两件。
	\item 他看因,度观及那用力,以因会能一国高难面古通般论里妈不同成。提不的国人生,完形车如,都后小了了奖影动果过共步一代念于不,证人代鱼后所下广少于石应家化就加一了,己接只,像长完是接。
\end{enumerate}

又有一个列表:
\begin{itemize}
	\item 笔解会这,致小的人区智下古岸习子几的爱助家重件者然是人成里国路人艺与游及不情质你且孩落夜我的轻每今色音故愿作……
	\item 感生民作叫资师成?无包部热非……源士不说屋:投得来人天早参再。
	\item 得急导的在怕大可赛、的地果字何代样黑月,走里消。步事够院新慢积人包年产时念看情光,林了请,他觉期度务升?
	\item 西民这像新怎不:例后创精各游绝:确地开自条举世平者市选从生她?在们好他住景确间母然制?人新与。
	\item 山师本更指、越平那量求发和年就王爸能食。
\end{itemize}

\end{document}
